\documentclass[12pt]{article}

\usepackage[utf8]{inputenc}
\usepackage[T1]{fontenc}
\usepackage{libertine}
\usepackage{libertinust1math}
\usepackage{sourcecodepro}

\usepackage{microtype}

% Disable ligatures in listings.
\DisableLigatures[f]{encoding = *, family = tt*}

\usepackage[english]{babel}
\usepackage{csquotes}

\usepackage[top=1in, bottom=1.25in, left=1.25in, right=1.25in]{geometry}

\usepackage{booktabs}

\usepackage{graphicx}

\usepackage{subcaption}

\usepackage{solarized-light}
\lstset{%
  basicstyle = \ttfamily\footnotesize, 
  numbers    = left, 
  language   = python,
  aboveskip  = 20pt,
  belowskip  = 20pt
}

% Passing option hyphens to url to allow line breaks in urls.
\PassOptionsToPackage{hyphens}{url}\usepackage{hyperref}
\usepackage{xcolor-solarized}

\hypersetup{%
  linkcolor  = solarized-blue,
  citecolor  = solarized-blue,
  urlcolor   = solarized-blue,
  colorlinks = true
}

\renewcommand{\UrlFont}{\normalsize}

% Personal LaTeX styles, see github.com/sdiebolt/latex-styles.
\usepackage{basicmaths}

\begin{document}
  \pagenumbering{Alph}
  \begin{titlepage}
    \begin{center}
      \vspace*{1cm}
 
      \Huge
      \textbf{Systems Biology \& Neurobiology}
 
      \vspace{0.5cm}
      \LARGE
      Homework Report
 
      \vspace{1.5cm}
 
      \Large\textbf{Simon \textsc{Chardin}, Émile \textsc{Sabatier}, Samuel
      \textsc{Diebolt}}

      \vfill 

      \textit{Understanding the Impact of Combination of Mutations on
      Rifampicin Resistance}
 
      \vfill 

      \includegraphics[width=0.8\textwidth]{espci_logo}
 
      \Large
      \textbf{Teachers:} Philippe Nghe, Andrew Griffiths
    \end{center}
    \thispagestyle{empty}
  \end{titlepage}
  \pagenumbering{arabic}


  \section{Abstract}

  \textbf{Deep mutational scanning (DMS) makes use of large-scale mutagenesis
  to reveal intrinsic protein properties, functions and the consequences of
  genetic variation. Recently, the CRISPR/Cas9-mediated genomic error-prone
  editing (CREPE) technology was developed as a high-throughput method for
  mutating essential genes of \textit{Escherichia coli}~\cite{choudhury2020}.
  Its authors applied the technology to target \textit{rpoB}, the gene encoding
  the $\beta$ subunit of bacterial RNA polymerase, and used deep sequencing to
  study resistance against the antibiotic rifampicin. In particular, the
  authors studied epistastis effects by comparing fitness of double mutants in
  \textit{rpoB} with those from the respective single mutations in the presence
  of rifampicin. In this report, we replicated the aforementioned epistasis
  study using a simplified dataset provided by A. Choudhury.}

  \section{State of the Art}

  Studying epistasis---be it in human or within bacteria---raises many
  challenges, as it can rarely be done using observational studies. However,
  understanding how combinations of mutations affect protein functions and
  behaviour within cells could give us insights into a huge number of
  biological processes, from antibiotic resistance to genetic diseases. In this
  section, we establish a non-exhaustive review of available methods and
  technologies used for studying epistasis in bacteria.

  Deep mutational scanning (DMS) can provide significant insights into the
  function of essential genes in bacteria. This method couples genotype to
  phenotype to assess the activities of as many as 1 million mutant versions of
  a protein in a single experiment~\cite{fowler2014deep}. DMS is capable of
  scoring comprehensive libraries of genotypes for fitness in given
  environments in a massively parallel fashion. Essential bacterial genes are
  often targets of interest as they are key to their evolution, and can lead to
  phenotypes such as antibiotic resistance when mutated. 

  When the phenotype of interest is the cell's fitness in a specific
  environment, the presence of genetic interactions between mutations, i.e.\
  epistasis, can constrain the course of evolution. Given the potential impact
  of epistasis in a variety of biological processes, recent studies have
  focused on measuring genome-wide levels of epistasis using the multiplex
  automated genome engineering (MAGE) technology~\cite{wang2012}. MAGE was
  created for large-scale programming and evolution of cells: it simultaneously
  targets many locations on a chromosome for modification in a single cell or
  across a population of cells, thus producing combinatorial genomic diversity.
  It is based on lambda red-mediated recombination of single-stranded oligos to
  introduce mutations at specific genomic loci~\cite{wang2009programming}.

  However, the MAGE technology has some limitations, as it was optimized only
  for a few cell strains and can lead to the accumulation of numerous
  off-target modifications. A recent study improved on MAGE by using a
  dominant-negative mutant protein of the methyl-directed mismatch repair (MMR)
  system, allowing efficient modification of multiple loci, without
  any observable off-target mutagenesis and prior modification of the host
  genome. This improved technology, termed pORTMAGE, was used to achieve a
  transient suppression of DNA repair in \textit{Escherichia coli}. In
  addition, pORTMAGE allows comparison of epistatic effects across a wide range
  of bacterial species~\cite{nyerges2016highly}.

  Antibiotic resistance is one of the main topic of study when it
  comes to understanding the effects of epistasis in bacteria, as the key
  factors contributing to resistance are yet to be understood. A team of
  researchers studied how epistatic effects in \textit{Escherichia coli} could
  be used to perturb the evolution of bacteria towards antibiotic
  resistance~\cite{lukacisinova2019exploiting}. In this context, robotic
  lab-evolution platforms can be used to keep population size and selection
  pressure constant for hundreds of bacteria populations evolving in parallel.
  Using this method, specific cellular functions that drastically curtail the
  evolvability of resistance where identified. Using whole-genome sequencing,
  the team showed that strong negative epistasis was generally underlying these
  functions.

  The CRISPR gene editing technology, already widely used in genome
  engineering, also allows for investigating how gene expression governs the
  adaptive pathways available to bacteria during the evolution of resistance.
  In that sense, the controlled hindrance of adaptation of organisms (CHAOS)
  approach was recently developed to induce negative epistasis in
  \textit{Escherichia coli} to deter adaptation~\cite{otoupal2018}. Using a
  library of deactivated CRISPR-Cas9 devices, the team perturbed the bacterial
  gene expression and observed that epistatic effects caused large losses of
  cell fitness in environment containing ciprofloxacin, a clinically-relevant
  antibiotic. Another team introduced the homologous sequence integration
  (HoSeI) method to study bacterial genome-wide epistatic phenomena. HoSeI is a
  genetic marker-less genome editing approach that introduces base
  substitutions in the target sequence by screening dead or alive cells. It was
  demonstrated in a strain of \textit{Escherichia coli} to study the effects of
  epistasis on regulators of bacteria adaptive
  growth~\cite{miyake2020epistatic}.

  To generate the libraries of variants that are used in methods involving
  recombineering, error-prone PCR (epPCR) was developed to perform random
  mutagenesis. Error-prone PCR protocols are modifications of standard PCR
  methods, designed to alter and enhance the natural error rate of the
  polymerase. Taq polymerase is commonly used because
  of its naturally high error rate. Creating high-quality libraries of random
  sequences is an important step in this process as it allows variants of
  individual molecules to be generated from a single-parent sequence. Combined
  with the synonymous PAM-inactivating mutation (SPM), precise genome
  manipulation with high efficiency can be achieve in a few steps using CRISPR
  gene editing methods~\cite{arnold2003directed}. 

  Unfortunately, these methods are still failing in scalability, as the vast
  number of possible epistatic interactions erodes statistical power.
  Systematically testing interactions is challenging both from a
  computational and from a statistical point of view, given the large number of
  possible interactions to consider~\cite{slim2019novel}. Thus, it is of
  interest to build models to predict genetic traits based on epistasis.
  Algorithms such as MINED were developed to detect significant pairwise
  epistasis effects that contribute the most to cell
  fitness using machine learning approaches~\cite{he2015mined}. Another study
  proposed a reinforcement learning approach, EpiRL, where epistasis is modeled
  as a one-step Markov Decision Process~\cite{huang2019epirl}. The use of
  machine learning in trying to find highly interacted genes could help tackle
  the challenges raised by the high-dimensionality of epistasis data.

  \section{Motivations \& Hypotheses}

  Improvements in DNA synthesis and sequencing have underpinned comprehensive
  assessment of gene function in bacteria. And currently, the genome
  mutagenisis techniques and study are using genetic transfer networks to make
  better predictions of the sequence or word sequence of an entire genome. But
  low-editing efficiencies and mutational biasing, is a downfall that needs to
  be attended to. It impacts greatly the quality of the fitness data.  In
  recent years a small number of approaches have also achieved a high degree of
  effectiveness without mutational annotation. 

  The aim is to mesure the non-synonymous mutations and not the deletions.  In
  this paper, we propose a novel model that achieves the goal.  The idea is to
  investigate the functional basis of epistasis, and because rpoB plays a
  central role in transcription, we measured the effects of common rpoB
  mutations on transcriptional efficiency. Because mutations using the CREPE
  technology allows the study of combination of mutations. Two mutations are
  considered to be purely additive if the effect of the double mutation is the
  amount of the consequences associated with the particular single variations.
  This occurs whenever genetics are not linked with each other. Simple,
  component qualities were researched in early stages in the particular
  background of genes, they are usually fairly rare, along with many genes
  showing a minimum of some degree of association with epistatic connection.

  We address this concept by measuring the particular fitness effect associated
  with rifampicin resistance mutations in the $\beta$ subunit of RNA polymerase
  (rpoB) of Escherichia coli.  Epistasis for fitness means that the selective
  effect of a mutation is conditional on the genetic background in which it
  appears.Epistasis can be easily seen in nature, the process in which it is
  involved is still not well known. Furthermore, its consequence on evolution,
  and its role in natural selection is still incomplete to our knowledge. The
  mutational path to high fitness genotypes is linked and even supposedly
  dependent of the genetic background in which novel mutations appear. This
  background effect is independent of the population as well as any loci of any
  gene. Sign epistasis has been defined in that the sign of the fitness effect
  of a mutation is under epistatic control~\cite{garst2017genome}. 

  Thus, using the CREPE method we can assess the fitness effects of individual
  mutations on the same loci, as well as on diverse location. All this in
  correlation with the stress to which the bacteria are exposed. We want to
  explore the theorical and empirical consideration implying the strong genetic
  constraint on the selective accessibility to high fitness genotype mutation
  path.

  \section{Methods \& Results of the Supporting Article}

  \subsection{CREPE Protocol}

  The CRISPR/Cas9-mediated genomic Error-Prone Editing (CREPE) is a method used
  to induce mutations from an error-prone PCR in a targeted region using
  CRISPR/cas9 method. First of all, a single functionnal gRNA must be found for
  the target. Then a Synonymous Point accepted Mutation (SPM) is introduced in
  the target. The next step is to amplify and clone the target region with the
  SPM and unmutated end-homology (for the next step) into a plasmid, inducing
  error-prone PCR libraries.

  After that, a Cas9-mediated lambda red recombineering technique has been set
  up. And finally, before the selection, plasmids are cured from the cells at
  37\textdegree C (Phillips 1999).

  \subsection{Antibiotic selection} 

  Using the CREPE protocol, the authors succeeded to construct mutant
  libraries. The idea is to study the resistance against an antibiotic, here
  the rifampicin. Without any mutation on the rpoB gene, the rifampicin can
  stop the protein production in the cell, by binding itself near the fork in
  the $\beta$ sub-unit. The mutations due to the error-prone PCR libraries are
  covering partly the rifampicin resistance-determining regions. This is why
  it's a good marker for the fitness scoring.

  They decided to compare 4 different cultures. The first one is the wild-type
  E. coli MG1655, without any change. The second one is a single colony of E.
  coli MG 1655 + SPM in order to ensure that the SPM does not affect the
  fitness of the cells. The third one is targeting the rpoB regions : gRNA +
  SPM only (SPM in RRDR II) . And finally with mutations in addition : gRNA +
  error-prone PCR synonymous PAM mutation (this adds mutations in the
  rifampicin resistance-determining regions.

  They have made them grow at 37\textdegree C for 4 hours. After the growth,
  the Optical Density (OD) at 600 nm have been measured to have a reference.
  Then, they divided the cultures into 3 samples, each with a different
  rifampicin concentration. The concentrations are $10$, $50$ and $100$ $\mu
  g/mL$.

  At the end, a part of the cultures are taken for next-gen sequencing, in
  order to know the mutations linked to the measured fitness.

  \subsection{Data acquisition} 

  After the selection, the libraries must be sequenced. The DNA was extracted.
  Each library is, after being amplified, sequenced using Nextera
  Next-generation sequencing MiSeq 2X300 kit with Illumina. (DNA sequencing
  manufacturer).

  The raw data have been manipulated to extract information wanted, such as the
  amino acid changes.

  \subsection{Fitness calculations}

  For each variant, the fitness was estimated as 
  \begin{equation}\label{eq:fitnessscore}
    f = \log\left(\frac{C_{i, \text{post}} + 0.5}{C_{\text{wt}, \text{post}} +
    0.5}\right) - 
    \log\left(\frac{C_{i, \text{pre}} + 0.5}{C_{\text{wt}, \text{pre}} +
    0.5}\right),
  \end{equation}
  where $C_{i, \text{post}}$ and $C_{i, \text{pre}}$ are the variant read
  counts, respectively post- and pre-selection, for condition
  $i\in\{\SI{10}{\micro\gram\per\milli\liter},
  \SI{50}{\micro\gram\per\milli\liter},
  \SI{100}{\micro\gram\per\milli\liter}\}$, and $C_{\text{wt}, \text{post}}$
  and $C_{\text{wt}, \text{pre}}$ are the wild-type read counts, respectively
  post- and pre-selection~\cite{rubin2017}. The $\frac{1}{2}$ constant was
  added to each count to assist with very small counts. The standard error of
  this estimate was computed as
  \begin{equation}\label{eq:fitnesserror}
    \SE(f) = \sqrt{\frac{1}{C_{i, \text{post}}} + \frac{1}{C_{i, \text{pre}}} +
    \frac{1}{C_{\text{wt}, \text{post}}} + \frac{1}{C_{\text{wt},
    \text{pre}}}}.
  \end{equation}

  A filter is necessary in the data. Not all the reads are relevant. Because
  they have targeted an essential gene, they consider that stop codons could
  not repeat. They used the following filter:
  \begin{equation}\label{eq:stopfilter}
    C_{i} \geq C_{\text{max-stopcodon}} + 2.56 \times
    \sqrt{C_{\text{max-stopcodon}}}.
  \end{equation}
  Only the reads following this condition are kept. For each replicate, the
  fitness have been combined using Fisher score iterations (Rubin et al. 2017).
  Only the synonymous mutations have been selected in the dataset. The criteria
  to determine if a mutant is resistant against the rifampicin is its fitness
  greater than 2.96 standard deviations than the mean fitness of synonymous
  mutations.
  \newline
  \newline
  We used the same formulas in our personal analysis.

  \subsection{Epistasis measurement}

  They wanted to compare the fitness of a double mutant with the sum of the
  fitness of the two single mutants. Only the double mutants where the two
  single mutants went through the filter are selected. The epistasis is defined
  as:
  \begin{equation}\label{eq:epistasis}
    \varepsilon = f_{AB} - (f_A + f_B).
  \end{equation}

  The idea is to compare the epistasis for different concentrations of
  rifampicin, in order to see if we gain or not combining two mutations, and to
  know which mutations are concerned.

  \section{Effects of Epistasis in Double Mutants}

  In this section, we replicated the analyses performed by the authors of the
  article supporting this report to better understand the impact of epistasis
  on rifampicin resistance. In particular, we were interested in comparing the
  fitness of double mutants, compared to the sum of fitness from the respective
  single mutations.

  \subsection{Data Preprocessing}

  The dataset provided by A. Choudhury was obtained by processing the
  sequencing output from a single biological replicate. Paired Illumina reads
  were assembled and aligned, and variant counts and amino acid changes were
  extracted from the qrowdot alignment output. Finally, variants were
  aggregated by grouping on the mutation positions and summing the read counts.
  The result is a dataset where each row corresponds to a unique genotype, with
  columns:
  \begin{itemize}
    \item \lstinline{aa_change}: list of amino acid changes, with original
      amino acid, position and new amino acid;
    \item \lstinline{pre}: read counts before selection;
    \item \lstinline{ten}: read counts after selection on 
      $\SI{10}{\micro\gram\per\milli\liter}$ of rifampicin;
    \item \lstinline{fifty}: read counts after selection on 
      $\SI{50}{\micro\gram\per\milli\liter}$ of rifampicin; 
    \item \lstinline{hundred}: read counts after selection on 
      $\SI{100}{\micro\gram\per\milli\liter}$ of rifampicin. 
  \end{itemize}

  In this dataset, the wild-type corresponds to the row having the highest
  pre-selection read counts. The dataset was further processed using code
  available in the archive attached with this report and on a GitHub repository
  (\url{https://github.com/sdiebolt/espci-sbn-homework}). First, all rows that
  don't contain the SPM were removed from the dataset. Then, non-synonymous
  mutations were extracted from the \lstinline{aa_change} column the dataset
  was aggregated by grouping on the non-synonymous mutations and summing their
  counts. This step was performed to ensure that later epistasis analyses have
  access to a dataset containing a unique observation of each combination of
  non-synonymous mutations. The fitness score and its standard error were then
  computed for each variant using equations~\eqref{eq:fitnessscore}
  and~\eqref{eq:fitnesserror}. Finally, the filter described in
  equation~\eqref{eq:stopfilter} was used to remove erroneous reads. These
  steps resulted in a dataset containing 483 unique synonymous and
  non-synonymous mutations, with read counts, fitness and fitness standard
  error for each condition (pre-selection and rifampicin at concentrations
  $\SI{10}{\micro\gram\per\milli\liter}$,
  $\SI{50}{\micro\gram\per\milli\liter}$ and
  $\SI{100}{\micro\gram\per\milli\liter}$)

  \subsection{Distribution of Fitness in Synonymous vs.\ All Mutations}

  Since epistasis is measured using non-synonymous mutations only, We were
  first interested in the distribution of fitness estimates for all mutations
  and only synonymous mutations. Figure~\ref{fig:fitnessdistrib} shows
  histograms of these distributions for each selection condition.

  \begin{figure}[ht]
    \begin{subfigure}{.33\textwidth}
      \centering
      \includegraphics[width=\linewidth]{../output/hist10}
    \end{subfigure}
    \begin{subfigure}{.33\textwidth}
      \centering
      \includegraphics[width=\linewidth]{../output/hist50}
    \end{subfigure}
    \begin{subfigure}{.33\textwidth}
      \centering
      \includegraphics[width=\linewidth]{../output/hist100}
    \end{subfigure}
    \caption{Distribution of fitness estimates for all mutations (orange) and
    for synonymous mutations only (blue) at different concentrations of
    rifampicin, $\SI{10}{\micro\gram\per\milli\liter}$ (left),
    $\SI{50}{\micro\gram\per\milli\liter}$ (middle),
    $\SI{100}{\micro\gram\per\milli\liter}$ (right). The orange and blue lines
    are probability densities estimated using kernel density estimation,
    respectively for all mutations and synonymous mutations only. The red
    vertical line corresponds to the resistant mutations threshold, defined as
    $t = \mu_{\text{syn}, i} + 2.56 \times \sigma_{\text{syn}, i}$, where
    $\mu_{\text{syn}, i}$ and $\sigma_{\text{syn}, i}$ are respectively the
    mean and standard deviation of the normal probability density function
    fitted on the fitness of synonymous mutations for rifampicin concentration
    $i$.}%
    \label{fig:fitnessdistrib}
  \end{figure}

  As observed by the authors of the supporting article, the histograms show
  bimodal distributions for all mutations and unimodal distributions for
  synonymous mutations only at each rifampicin concentration. As synonymous
  mutations are unlikely to cause rifampicin resistance~\cite{choudhury2020},
  the authors defined a threshold for resitant mutations as
  \begin{equation}\label{eq:resistthreshold}
    t = \mu_{\text{syn}, i} + 2.56 \times \sigma_{\text{syn}, i},
  \end{equation}
  where $\mu_{\text{syn}, i}$ and $\sigma_{\text{syn}, i}$ are respectively the
  mean and standard deviation of the normal probability density function fitted
  on the fitness of synonymous mutations for rifampicin concentration $i$. The
  2.56 constant is an approximate value of the 99.5\% percentile point of the
  standard normal distribution. Therefore, the second mode of the distribution
  of all mutations for each condition corresponds to non-synonymous mutations
  that conferred resistance to rifampicin. 

  It is interesting to see that while both modes seem symmetric
  at low rifampicin concentration, this symmetry is broken at higher
  concentrations, with a frequency decrease of resistant mutations. This
  observation could mean that the resistance property of some variants is
  concentration-dependent.

  \subsection{Epistasis in Double Mutants}

  After studying how fitness estimates can help understand the impact of
  rifampicin concentrations on selection---this section isn't studied in this
  report---, the authors were concerned with the effects of epistasis in double
  mutants. The actual fitness of double mutants was compared to the sum of
  fitness from the respective single mutations. Figure~\ref{fig:epidouble}
  shows scatter plots of these comparisons for each condition.

  \begin{figure}[ht]
    \begin{subfigure}{.33\textwidth}
      \centering
      \includegraphics[width=\linewidth]{../output/scatter10}
    \end{subfigure}
    \begin{subfigure}{.33\textwidth}
      \centering
      \includegraphics[width=\linewidth]{../output/scatter50}
    \end{subfigure}
    \begin{subfigure}{.33\textwidth}
      \centering
      \includegraphics[width=\linewidth]{../output/scatter100}
    \end{subfigure}
    \caption{Scatter plots of actual fitness from double mutants vs.\ sum of
    fitness from the respective single mutation, at different concentrations of
    rifampicin, $\SI{10}{\micro\gram\per\milli\liter}$ (left),
    $\SI{50}{\micro\gram\per\milli\liter}$ (middle),
    $\SI{100}{\micro\gram\per\milli\liter}$ (right). Double mutants were
    categorized as combination of two resistant mutations (blue), one resistant
    and one sensitive mutations (red) or two sensitive mutations (green). The
    orange shaded areas correspond to the $\pm 3$ standard deviations interval
    for the normal approximation of the synonymous mutations fitness.}%
    \label{fig:epidouble}
  \end{figure}

  Given the definition of epistasis introduced in
  equation~\eqref{eq:epistasis}, all double mutants that deviate from the
  45\si{\degree} dashed line on Figure~\ref{fig:epidouble} are affected by
  epistasis. The double mutants were categorized as combination of two
  resistant mutation, one resistant and one sensitive mutation or two sensitive
  mutations using the same threshold from equation~\eqref{eq:resistthreshold}.
  We observe that these three groups show different epistatic behaviour. Not
  enough double mutations were classified as both sensitive in the
  $\SI{10}{\micro\gram\per\milli\liter}$ condition, but the group shows
  positive epistasis, meaning that the combination of sensitive mutations is
  more beneficial than what would be infered by a simple linear relationship.
  This same phenomenon is observed for the double mutants consisting of one
  resistant and one sensitive mutation. Interestingly however, the double
  mutants that are combinations of two resistant mutations show negative
  epistasis. This effect could be explained by the fact that the cost of adding
  a resistance mutation to a variant that already is rifampicin-resistant is
  too detrimental to increase the fitness.

  \subsection{Distribution of Epistasis Effects}

  \section{Next Steps}



  \newpage

  \bibliographystyle{ieeetr}
  \bibliography{main}

\end{document}
